% allgemeine Formatangaben
\documentclass[
 a4paper, 										% Papierformat
 12pt,											% Schriftgröße
 ngerman, 										% für Umlaute, Silbentrennung etc.
 titlepage,										% es wird eine Titelseite verwendet
 bibliography=totoc,							% Literaturverzeichnis im Inhaltsverzeichnis aufführen
 listof=totoc,									% Verzeichnisse im Inhaltsverzeichnis aufführen
 oneside, 										% einseitiges Dokument
 captions=nooneline,							% einzeilige Gleitobjekttitel ohne Sonderbehandlung wie mehrzeilige Gleitobjekttitel behandeln
 numbers=noenddot,								% Überschriften-Nummerierung ohne Punkt am Ende
 parskip=half									% zwischen Absätzen wird eine halbe Zeile eingefügt
 ]{scrbook}

% Anpassung an Landessprache
\usepackage[ngerman]{babel}
 
% Verwenden von Sonderzeichen und Silbentrennung
\usepackage[utf8]{inputenc}	% Verwendung des Unicode
\usepackage[T1]{fontenc}											%
\usepackage{textcomp} 											% Euro-Zeichen und andere
\usepackage[babel,german=quotes]{csquotes}							% Anf\"uhrungszeichen
\RequirePackage[ngerman=ngerman-x-latest]{hyphsubst} 				% erweiterte Silbentrennung

% Befehle aus AMSTeX f\"ur mathematische Symbole z.B. \boldsymbol \mathbb
\usepackage{amsmath,amsfonts}

% Test
\usepackage{etex}
\usepackage[justification=centering]{caption}
\usepackage{hyperref}

% Zeilenabstände und Seitenränder 
\usepackage{setspace}
\usepackage{geometry}

% Einbinden von JPG-Grafiken
\usepackage{graphicx}

% zum Umflie�{\ss}en von Bildern
% Verwendung unter http://de.wikibooks.org/wiki/LaTeX-Kompendium:_Baukastensystem#textumflossene_Bilder
\usepackage{floatflt}
\usepackage{float}

% Verwendung von vordefinierten Farbnamen zur Colorierung
% Palette und Verwendung unter http://kitt.cl.uzh.ch/kitt/CLinZ.CH/src/Kurse/archiv/LaTeX-Kurs-Farben.pdf
\usepackage[usenames,dvipsnames]{color} 

% Tabellen
\usepackage{array}
\usepackage{longtable}

% einfache Grafiken im Code
% Einführung unter http://www.math.uni-rostock.de/~dittmer/bsp/pstricks-bsp.pdf
\usepackage{pstricks}



% Quellcodeansichten
\usepackage{verbatim}
\usepackage{moreverb} 											% für erweiterte Optionen der verbatim Umgebung
% Befehle und Beispiele unter http://www.ctex.org/documents/packages/verbatim/moreverb.pdf
\usepackage{listings} 											% für angepasste Quellcodeansichten siehe
% Kurzeinführung unter http://blog.robert-kummer.de/2006/04/latex-quellcode-listing.html

% Glossar und Abbildungsverzeichnis
\usepackage[
nonumberlist,		% keine Seitenzahlen anzeigen
acronym,			% ein Abkürzungsverzeichnis erstellen
toc					% Einträge im Inhaltsverzeichnis
]					% im Inhaltsverzeichnis auf section-Ebene erscheinen
{glossaries}

% verlinktes und Farblich angepasstes Inhaltsverzeichnis
% \usepackage[
% pdftex,
% colorlinks=true,
% linkcolor=InterneLinkfarbe,
% urlcolor=ExterneLinkfarbe,
% citecolor=InterneLinkfarbe
% ]{hyperref}
% \usepackage[all]{hypcap}

% URL verlinken, lange URLs umbrechen
\usepackage{url}

% sorgt dafür, dass Leerzeichen hinter parameterlosen Makros nicht als Makroendezeichen interpretiert werden
\usepackage{xspace}

% Beschriftungen f\"ur Abbildungen und Tabellen
\usepackage{caption}

% Entwicklerwarnmeldungen entfernen
\usepackage{scrhack}

% Tabelle mit geeigneter Breite
\usepackage{tabularx}


\usepackage{wrapfig}
% Alternative zu bibtex (unterstützt Webseitenzitate)
%\usepackage{biblatex}

% Erweitrung zu bibtex (unterstützt weitere Elemente)
%\usepackage[authoryear]{natbib}							% einbinden der verwendeten Latex-Pakete
\newcommand{\hochschule}{Hochschule für Technik, Wirtschaft und Kultur}
\newcommand{\fachbereich}{Fakultät Informatik, Mathematik und Naturwissenschaften}
\newcommand{\autor}{Artem Pokas}
\newcommand{\ort}{Leipzig}
 						% einbinden von persönlichen Daten

\onehalfspacing 								% 1,5-facher Zeilenabstand

\definecolor{InterneLinkfarbe}{rgb}{0.1,0.1,0.3} 	% Farbliche Absetzung von externen Links
\definecolor{ExterneLinkfarbe}{rgb}{0.1,0.1,0.7}	% Farbliche Absetzung von internen Links

% Einstellungen für Fußnoten:
\captionsetup{font=footnotesize,labelfont=sc,singlelinecheck=true,margin={5mm,5mm}}

% Stil der Quellenangabe
\bibliographystyle{plain}
%\bibliographystyle{alphadin}
%\bibliographystyle{apalike}



%Ausschluss von Schusterjungen
\clubpenalty = 10000
%Ausschluss von Hurenkindern
\widowpenalty = 10000

% Befehle, die Umlaute ausgeben, führen zu Fehlern, wenn sie hyperref als Optionen übergeben werden
\hypersetup{
    pdftitle={\titel \untertitel},
    pdfauthor={\autor},
    pdfcreator={\autor},
    pdfsubject={\titel \untertitel},
    pdfkeywords={\titel \untertitel},
}

% Listings
\lstloadlanguages{Java,HTML}

% Bessere dickengleiche Schrift.
\renewcommand{\ttdefault}{pcr}

% Farben
\definecolor{colorcomment}{rgb}{0.1,0.1,0.62}

% List
\lstset{
	frame=tb,
	framesep=5pt,
	basicstyle=\footnotesize\ttfamily,
	showstringspaces=false,
	keywordstyle=\ttfamily\bfseries\color{CadetBlue},
	identifierstyle=\ttfamily,
	stringstyle=\ttfamily\color{OliveGreen},
	commentstyle=\color{colorcomment},
	rulecolor=\color{Gray},
	xleftmargin=5pt,
	xrightmargin=5pt,
	aboveskip=\bigskipamount,
	belowskip=\bigskipamount,
	tabsize=2,
	numbers=left,                    
	numbersep=5pt,                   
	numberstyle=\footnotesize\color{gray}, 	
	escapeinside={@}{@},          
	% caption=\lstname  
}

% Listings, die nicht \"uber mehrere Seiten gehen
\newcommand{\code}[2][]{
	\begin{minipage}[t]{\textwidth}
		\lstinputlisting[#1]{#2} 
	\end{minipage}
}

%Den Punkt am Ende jeder Beschreibung deaktivieren
\renewcommand*{\glspostdescription}{}
%Bild-Caption leer lassen
\addto\captionsngerman{\renewcommand{\figurename}{Bild}}

\renewcommand*{\partheadendvskip}{}


%Glossar-Befehle anschalten
\makeglossaries
\glsenablehyper
\newacronym{FOV}{FOV}{Field of View\protect\glsadd{glos:FOV}}
\newglossaryentry{glos:FOV}{
name=Field Of View,
description={Dies ist ein Glossareintrag.} 
}  
