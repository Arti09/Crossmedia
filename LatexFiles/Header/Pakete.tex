% Anpassung an Landessprache
\usepackage[ngerman]{babel}
 
% Verwenden von Sonderzeichen und Silbentrennung
\usepackage[utf8]{inputenc}	% Verwendung des Unicode
\usepackage[T1]{fontenc}											%
\usepackage{textcomp} 											% Euro-Zeichen und andere
\usepackage[babel,german=quotes]{csquotes}							% Anf\"uhrungszeichen
\RequirePackage[ngerman=ngerman-x-latest]{hyphsubst} 				% erweiterte Silbentrennung

% Befehle aus AMSTeX f\"ur mathematische Symbole z.B. \boldsymbol \mathbb
\usepackage{amsmath,amsfonts}

% Test
\usepackage{etex}
\usepackage[justification=centering]{caption}
\usepackage{hyperref}

% Zeilenabstände und Seitenränder 
\usepackage{setspace}
\usepackage{geometry}

% Einbinden von JPG-Grafiken
\usepackage{graphicx}

% zum Umflie�{\ss}en von Bildern
% Verwendung unter http://de.wikibooks.org/wiki/LaTeX-Kompendium:_Baukastensystem#textumflossene_Bilder
\usepackage{floatflt}
\usepackage{float}

% Verwendung von vordefinierten Farbnamen zur Colorierung
% Palette und Verwendung unter http://kitt.cl.uzh.ch/kitt/CLinZ.CH/src/Kurse/archiv/LaTeX-Kurs-Farben.pdf
\usepackage[usenames,dvipsnames]{color} 

% Tabellen
\usepackage{array}
\usepackage{longtable}

% einfache Grafiken im Code
% Einführung unter http://www.math.uni-rostock.de/~dittmer/bsp/pstricks-bsp.pdf
\usepackage{pstricks}



% Quellcodeansichten
\usepackage{verbatim}
\usepackage{moreverb} 											% für erweiterte Optionen der verbatim Umgebung
% Befehle und Beispiele unter http://www.ctex.org/documents/packages/verbatim/moreverb.pdf
\usepackage{listings} 											% für angepasste Quellcodeansichten siehe
% Kurzeinführung unter http://blog.robert-kummer.de/2006/04/latex-quellcode-listing.html

% Glossar und Abbildungsverzeichnis
\usepackage[
nonumberlist,		% keine Seitenzahlen anzeigen
acronym,			% ein Abkürzungsverzeichnis erstellen
toc					% Einträge im Inhaltsverzeichnis
]					% im Inhaltsverzeichnis auf section-Ebene erscheinen
{glossaries}

% verlinktes und Farblich angepasstes Inhaltsverzeichnis
% \usepackage[
% pdftex,
% colorlinks=true,
% linkcolor=InterneLinkfarbe,
% urlcolor=ExterneLinkfarbe,
% citecolor=InterneLinkfarbe
% ]{hyperref}
% \usepackage[all]{hypcap}

% URL verlinken, lange URLs umbrechen
\usepackage{url}

% sorgt dafür, dass Leerzeichen hinter parameterlosen Makros nicht als Makroendezeichen interpretiert werden
\usepackage{xspace}

% Beschriftungen f\"ur Abbildungen und Tabellen
\usepackage{caption}

% Entwicklerwarnmeldungen entfernen
\usepackage{scrhack}

% Tabelle mit geeigneter Breite
\usepackage{tabularx}


\usepackage{wrapfig}
% Alternative zu bibtex (unterstützt Webseitenzitate)
%\usepackage{biblatex}

% Erweitrung zu bibtex (unterstützt weitere Elemente)
%\usepackage[authoryear]{natbib}